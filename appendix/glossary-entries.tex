% Glossary entries
\newglossaryentry{api} {
    name=\glslink{api}{API},
    text=Application Program Interface,
    sort=api,
    description={in informatica con il termine \emph{Application Programming Interface API} (ing. interfaccia di programmazione di un'applicazione) si indica ogni insieme di procedure disponibili al programmatore, di solito raggruppate a formare un set di strumenti specifici per l'espletamento di un determinato compito all'interno di un certo programma. La finalità è ottenere un'astrazione, di solito tra l'hardware e il programmatore o tra software a basso e quello ad alto livello semplificando così il lavoro di programmazione}
}

\newglossaryentry{poc} {
    name=\glslink{poc}{POC},
    text=Proof of Concept,
    description={La Proof of Concept (POC), in italiano "prova di concetto", è una realizzazione preliminare di un'idea o di una teoria per dimostrarne la fattibilità. È utilizzata in diversi campi, tra cui tecnologia, ingegneria, ricerca scientifica e sviluppo di prodotti, per verificare se un concetto o una proposta ha il potenziale per essere sviluppato ulteriormente e diventare un prodotto o un servizio funzionante.}
}

\newglossaryentry{clinical-decision-rules} {
    name=\glslink{clinical-decision-rules}{clinical decision rules},
    text=Clinical decision rules,
    description={Clinical decision rules (CDRs) sono strumenti utilizzati dai medici per migliorare e standardizzare il processo decisionale clinico. Queste regole sono derivate da studi di ricerca clinica e utilizzano informazioni cliniche specifiche per fornire raccomandazioni su come gestire un paziente. Le CDRs combinano diverse variabili cliniche per prevedere la probabilità di una determinata malattia o esito e forniscono una guida su quali azioni intraprendere, come ulteriori test diagnostici o trattamenti. Esempi noti di CDRs includono le regole di Ottawa per determinare se un paziente con un trauma alla caviglia necessita di radiografie.}
}

\newglossaryentry{clinical-prediction-rules} {
    name=\glslink{clinical-prediction-rules}{clinical prediction rules},
    text=Clinical prediction rules,
    description={Clinical prediction rules (CPRs) sono simili alle CDRs, ma sono specificamente focalizzate sulla previsione di esiti clinici. Le CPRs utilizzano una combinazione di segni, sintomi e risultati di test per calcolare la probabilità che un paziente sviluppi un determinato esito clinico, come la probabilità di una diagnosi, il rischio di complicazioni, o la risposta a un trattamento. Le CPRs sono strumenti quantitativi che possono aiutare i medici a stratificare i pazienti in base al rischio e a prendere decisioni più informate. Un esempio è il punteggio Wells per la previsione della probabilità di trombosi venosa profonda.}
}

\newglossaryentry{point-of-care} {
    name=\glslink{point-of-care}{point-of-care},
    text=Point-of-care,
    description={Point-of-care (POC) si riferisce a qualsiasi servizio medico o test diagnostico effettuato al momento e nel luogo del trattamento del paziente. In altre parole, i test POC sono eseguiti vicino al paziente, spesso durante la visita medica, piuttosto che in un laboratorio remoto. Questo approccio consente di ottenere risultati rapidi e di prendere decisioni terapeutiche immediate. Esempi comuni di test POC includono test di gravidanza, glucometri per il controllo del glucosio nel sangue, e test rapidi per l'influenza o il COVID-19. L'obiettivo del POC è migliorare l'efficienza del trattamento, ridurre i tempi di attesa per i risultati e migliorare la qualità delle cure fornite}
}