% \omiss produces '[...]'
\newcommand{\omissis}{[\dots\negthinspace]}

% Itemize symbols
% see: https://tex.stackexchange.com/a/62497
% \renewcommand{\labelitemi}{$\bullet$}
% \renewcommand{\labelitemii}{$\cdot$}
% \renewcommand{\labelitemiii}{$\diamond$}
% \renewcommand{\labelitemiv}{$\ast$}


\let\Chaptermark\chaptermark
% \Chaptername gives current chapter name
\def\chaptermark#1{\def\Chaptername{#1}\Chaptermark{#1}}
\makeatletter
% \currentname gives the current section name
\newcommand*{\currentname}{\@currentlabelname}
\makeatother

% Uncomment the following line for a different header/footer style
% \pagestyle{fancy} \setlength{\headheight}{14.5pt}
\fancyhead[L]{\fontsize{12}{14.5} \selectfont \thechapter. \Chaptername}
\fancyhead[R]{\fontsize{12}{14.5} \selectfont \currentname}
% Page number always in footer
\cfoot{\thepage}


% Custom hyphenation rules
\hyphenation {
    e-sem-pio
    ex-am-ple
}

% Images path, not using \graphicspath because it doesn't properly work with
% latexmk custom dependencies
\NewCommandCopy{\latexincludegraphics}{\includegraphics}
\renewcommand{\includegraphics}[2][]{\latexincludegraphics[#1]{../images/#2}}

% Page format settings
% see: http://wwwcdf.pd.infn.it/AppuntiLinux/a2547.htm
\setlength{\parindent}{14pt}    % first row indentation
\setlength{\parskip}{0pt}       % paragraphs spacing


% Load variables
\newcommand{\myName}{Giorgia Paparazzo}
\newcommand{\myID}{888319}
\newcommand{\myTitle}{Ottimizzazione e Scalabilità: reingegnerizzazione di una web application per il supporto delle decisioni in ambito ospedaliero}
\newcommand{\myDegree}{Tesi di Laurea Magistrale}
\newcommand{\myUni}{Università degli Studi di Milano Bicocca}
\newcommand{\myDepartment}{Dipartimento di Informatica, Sistemistica e Comunicazione}
% For BSc level just use "Corso di Laurea" and don't add "Triennale" to it
\newcommand{\myFaculty}{Corso di Laurea in Teoria e Tecnologia della Comunicazione}
\newcommand{\profTitle}{Prof.}
\newcommand{\myProf}{Federico Cabitza}
\newcommand{\coProf}{Andrea Campagner}
\newcommand{\myLocation}{Milano}
\newcommand{\myAA}{2023-2024}
\newcommand{\myTime}{Luglio 2024}

% PDF file metadata fields
% when updating them delete the build directory, otherwise they won't change
\begin{filecontents*}{\jobname.xmpdata}
  \Title{Ottimizzazione e Scalabilità: reingegnerizzazione di una web application per il supporto delle decisioni in ambito ospedaliero}
  \Author{Giorgia Paparazzo}
  \Language{it-IT}
  \Subject{Short description}
  \Keywords{incertezza\sep visualization\sep medicina}
\end{filecontents*}


% Glossary entries
\newglossaryentry{api} {
    name=\glslink{api}{API},
    text=Application Program Interface,
    sort=api,
    description={in informatica con il termine \emph{Application Programming Interface API} (ing. interfaccia di programmazione di un'applicazione) si indica ogni insieme di procedure disponibili al programmatore, di solito raggruppate a formare un set di strumenti specifici per l'espletamento di un determinato compito all'interno di un certo programma. La finalità è ottenere un'astrazione, di solito tra l'hardware e il programmatore o tra software a basso e quello ad alto livello semplificando così il lavoro di programmazione}
}

\newglossaryentry{poc} {
    name=\glslink{poc}{POC},
    text=Proof of Concept,
    description={La Proof of Concept (POC), in italiano "prova di concetto", è una realizzazione preliminare di un'idea o di una teoria per dimostrarne la fattibilità. È utilizzata in diversi campi, tra cui tecnologia, ingegneria, ricerca scientifica e sviluppo di prodotti, per verificare se un concetto o una proposta ha il potenziale per essere sviluppato ulteriormente e diventare un prodotto o un servizio funzionante.}
}

\newglossaryentry{clinical-decision-rules} {
    name=\glslink{clinical-decision-rules}{clinical decision rules},
    text=Clinical decision rules,
    description={Clinical decision rules (CDRs) sono strumenti utilizzati dai medici per migliorare e standardizzare il processo decisionale clinico. Queste regole sono derivate da studi di ricerca clinica e utilizzano informazioni cliniche specifiche per fornire raccomandazioni su come gestire un paziente. Le CDRs combinano diverse variabili cliniche per prevedere la probabilità di una determinata malattia o esito e forniscono una guida su quali azioni intraprendere, come ulteriori test diagnostici o trattamenti. Esempi noti di CDRs includono le regole di Ottawa per determinare se un paziente con un trauma alla caviglia necessita di radiografie.}
}

\newglossaryentry{clinical-prediction-rules} {
    name=\glslink{clinical-prediction-rules}{clinical prediction rules},
    text=Clinical prediction rules,
    description={Clinical prediction rules (CPRs) sono simili alle CDRs, ma sono specificamente focalizzate sulla previsione di esiti clinici. Le CPRs utilizzano una combinazione di segni, sintomi e risultati di test per calcolare la probabilità che un paziente sviluppi un determinato esito clinico, come la probabilità di una diagnosi, il rischio di complicazioni, o la risposta a un trattamento. Le CPRs sono strumenti quantitativi che possono aiutare i medici a stratificare i pazienti in base al rischio e a prendere decisioni più informate. Un esempio è il punteggio Wells per la previsione della probabilità di trombosi venosa profonda.}
}

\newglossaryentry{point-of-care} {
    name=\glslink{point-of-care}{point-of-care},
    text=Point-of-care,
    description={Point-of-care (POC) si riferisce a qualsiasi servizio medico o test diagnostico effettuato al momento e nel luogo del trattamento del paziente. In altre parole, i test POC sono eseguiti vicino al paziente, spesso durante la visita medica, piuttosto che in un laboratorio remoto. Questo approccio consente di ottenere risultati rapidi e di prendere decisioni terapeutiche immediate. Esempi comuni di test POC includono test di gravidanza, glucometri per il controllo del glucosio nel sangue, e test rapidi per l'influenza o il COVID-19. L'obiettivo del POC è migliorare l'efficienza del trattamento, ridurre i tempi di attesa per i risultati e migliorare la qualità delle cure fornite}
}

\newglossaryentry{test-del-tornasole} {
    name=\glslink{test-del-tornasole}{test del tornasole},
    text=test del tornasole,
    description={Scala di Colore: Le cartine tornasole sono dotate di una scala di colori che varia da 1 a 14. Questo permette di determinare con precisione il livello di pH di una sostanza. La scala si suddivide in: 1-3 per indicare forte acidità, 4-6 per acidità, 7 per neutralità, 8-11 per alcalinità e 12-14 per forte alcalinità.}
}

\newglossaryentry{quartili} {
    name=\glslink{quartili}{quartili},
    text=quartili,
    description={Una volta ordinati i dati, i quartili sono i tre valori che dividono l'insieme dei dati in quattro intervalli di uguale numerosità. Il secondo quartile coincide con la mediana della distribuzione. Per estensione di significato, si dice quartile anche ognuno dei quattro intervalli così determinati.}
}

\newglossaryentry{domain-driven-design} {
    name=\glslink{domain-driven-design}{domain driven design},
    text=Domain-Driven Design,
    description={È un approccio dello sviluppo del software che risolve problemi complessi connettendo l'implementazione ad un modello in evoluzione.}
}
\makeglossaries

\bibliography{appendix/bibliography}

\defbibheading{bibliography} {
    \cleardoublepage
    \phantomsection
    \addcontentsline{toc}{chapter}{\bibname}
    \chapter*{\bibname\markboth{\bibname}{\bibname}}
}

% Spacing between entries
\setlength\bibitemsep{1.5\itemsep}

\DeclareBibliographyCategory{opere}
\DeclareBibliographyCategory{web}

\addtocategory{opere}{womak:lean-thinking}
\addtocategory{web}{site:agile-manifesto}

\defbibheading{opere}{\section*{Riferimenti bibliografici}}
\defbibheading{web}{\section*{Siti Web consultati}}


\captionsetup{
    tableposition=top,
    figureposition=bottom,
    font=small,
    format=hang,
    labelfont=bf
}

\hypersetup{
    %hyperfootnotes=false,
    %pdfpagelabels,
    colorlinks=true,
    linktocpage=true,
    pdfstartpage=1,
    pdfstartview=,
    breaklinks=true,
    pdfpagemode=UseNone,
    pageanchor=true,
    pdfpagemode=UseOutlines,
    plainpages=false,
    bookmarksnumbered,
    bookmarksopen=true,
    bookmarksopenlevel=1,
    hypertexnames=true,
    pdfhighlight=/O,
    %nesting=true,
    %frenchlinks,
    urlcolor=webbrown,
    linkcolor=RoyalBlue,
    citecolor=webgreen
    %pagecolor=RoyalBlue,
}

% Delete all links and show them in black
\if \isprintable 1
    \hypersetup{draft}
\fi

% Listings setup
\lstset{
    language=[LaTeX]Tex,%C++,
    keywordstyle=\color{RoyalBlue}, %\bfseries,
    basicstyle=\small\ttfamily,
    %identifierstyle=\color{NavyBlue},
    commentstyle=\color{Green}\ttfamily,
    stringstyle=\rmfamily,
    numbers=none, %left,%
    numberstyle=\scriptsize, %\tiny
    stepnumber=5,
    numbersep=8pt,
    showstringspaces=false,
    breaklines=true,
    frameround=ftff,
    frame=single
}

\definecolor{webgreen}{rgb}{0,.5,0}
\definecolor{webbrown}{rgb}{.6,0,0}

\newcommand{\sectionname}{sezione}
\addto\captionsitalian{\renewcommand{\figurename}{Figura}
                       \renewcommand{\tablename}{Tabella}}

\newcommand{\glsfirstoccur}{\ap{{[g]}}}

\newcommand{\intro}[1]{\emph{\textsf{#1}}}

% Risks environment
\newcounter{riskcounter}                % define a counter
\setcounter{riskcounter}{0}             % set the counter to some initial value

%%%% Parameters
% #1: Title
\newenvironment{risk}[1]{
    \refstepcounter{riskcounter}        % increment counter
    \par \noindent                      % start new paragraph
    \textbf{\arabic{riskcounter}. #1}   % display the title before the content of the environment is displayed
}{
    \par\medskip
}

\newcommand{\riskname}{Rischio}

\newcommand{\riskdescription}[1]{\textbf{\\Descrizione:} #1.}

\newcommand{\risksolution}[1]{\textbf{\\Soluzione:} #1.}

% Use case environment
\newcounter{usecasecounter}             % define a counter
\setcounter{usecasecounter}{0}          % set the counter to some initial value

%%%% Parameters
% #1: ID
% #2: Nome
\newenvironment{usecase}[2]{
    \renewcommand{\theusecasecounter}{\usecasename #1}  % this is where the display of
                                                        % the counter is overwritten/modified
    \refstepcounter{usecasecounter}             % increment counter
    \vspace{10pt}
    \par \noindent                              % start new paragraph
    {\large \textbf{\usecasename #1: #2}}       % display the title before the
                                                % content of the environment is displayed
    \medskip
}{
    \medskip
}

\newcommand{\usecasename}{UC}

\newcommand{\usecaseactors}[1]{\textbf{\\Attori Principali:} #1. \vspace{4pt}}
\newcommand{\usecasepre}[1]{\textbf{\\Precondizioni:} #1. \vspace{4pt}}
\newcommand{\usecasedesc}[1]{\textbf{\\Descrizione:} #1. \vspace{4pt}}
\newcommand{\usecasepost}[1]{\textbf{\\Postcondizioni:} #1. \vspace{4pt}}
\newcommand{\usecasealt}[1]{\textbf{\\Scenario Alternativo:} #1. \vspace{4pt}}

% Namespace description environment
\newenvironment{namespacedesc}{
    \vspace{10pt}
    \par \noindent  % start new paragraph
    \begin{description}
}{
    \end{description}
    \medskip
}

\newcommand{\classdesc}[2]{\item[\textbf{#1:}] #2}
