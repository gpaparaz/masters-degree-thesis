\chapter{Visualizzazioni vaghe}
\label{cap:visualizzazioni-vaghe}

\section{Machine learning in ambito medico }

\section{Cosa ne pensano gli utenti nel 2024}
Secondo una recente indagine\footcite{womak:intelligenza-artificiale-e-medicina} dell'EngageMinds HUB, il Centro di ricerca dell'Università Cattolica, gli italiani ad oggi segnalano fiducia e timori verso l'uso dell'intelligenza artificiale in ambito medico.\\

Dal loro studio emerge che 6 italiani su 10 sono favorevoli all'uso dell'Intelligenza artificiale in ambito sanitario, di questi, l'88\% la userebbe per semplificare il linguaggio dei referti, l'86\% come supporto al medico per effettuare una diagnosi e l'80\% come aiuto per stabilire una terapia farmacologica adeguata, mentre quasi 6 italiani su 10 la utilizzerebbero come strumento per un'autoanalisi.\\

Di opinione meno positiva sono 7 italiani su 10 secondo i quali l'AI potrà causare una perdita della relazione e del contatto diretto con il medico.\\

Tra le principali opportunità che l'uso delle tecnologie digitali potranno portare, poco meno di 8 italiani su 10 (78\%) riferiscono che esse porteranno ad una maggiore accessibilità nell'accesso e nell'uso dei servizi, una riduzione dello spreco di carta e un maggior coinvolgimento del paziente grazie ad una maggiore accessibilità al proprio fascicolo sanitario. Il 74\% crede che le AI potranno ridurre i costi a lungo termine; poco meno di 7 su 10 ritengono che possa esserci un miglioramento nei monitoraggi tramite devices (68\%), mentre poco più di 6 su 10 si aspetta che le AI possano migliorare le diagnosi (63\%). Il 68\% degli italiani ritiene che l'uso di tecnologie digitali possano migliorare il monitoraggio da
remoto.\\

Un ulteriore rischio che gli italiani percepiscono è legato ai dati sensibili: per il 63\% l'uso dell'Intelligenza Artificiale potrà causare delle problematiche legate alla gestione della privacy, mentre per il 60\% legate alla diffusione di dati sensibili.




% \intro{Breve introduzione al capitolo}\\

% \section{Introduzione al progetto}

% \section{Analisi preventiva dei rischi}

% Durante la fase di analisi iniziale sono stati individuati alcuni possibili rischi a cui si potrà andare incontro.
% Si è quindi proceduto a elaborare delle possibili soluzioni per far fronte a tali rischi.\\

% \begin{risk}{Performance del simulatore hardware}
%     \riskdescription{le performance del simulatore hardware e la comunicazione con questo potrebbero risultare lenti o non abbastanza buoni da causare il fallimento dei test}
%     \risksolution{coinvolgimento del responsabile a capo del progetto relativo il simulatore hardware}
%     \label{risk:hardware-simulator} 
% \end{risk}

% \section{Requisiti e obiettivi}


% \section{Pianificazione}
