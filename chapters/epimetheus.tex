\chapter{Epimetheus}
\label{cap:epimetheus}

Il progetto Epimetheus nasce con l'obiettivo di migliorare le previsioni sullo stato di salute dei pazienti sottoposti a interventi chirurgici all'anca o al ginocchio. In un contesto medico sempre più orientato verso l'innovazione e la personalizzazione delle cure, Epimetheus si inserisce come una soluzione tecnologica avanzata che utilizza un complesso sistema di machine learning per elaborare predizioni a partire dai dati Patient-Reported Outcome Measures (PROM).\\

L'adozione di Epimetheus è prevista presso l'IRCCS Istituto Ortopedico Galeazzi (IOG) di Milano, un'importante struttura sanitaria universitaria specializzata nella diagnosi e nel trattamento dei disturbi muscolo-scheletrici. Presso lo IOG vengono eseguiti annualmente quasi 5000 interventi chirurgici, con una prevalenza di artroplastiche dell'anca e del ginocchio, oltre a numerose procedure relative alla colonna vertebrale.\\

Epimetheus mira a rispondere a una necessità cruciale: migliorare la qualità delle previsioni cliniche post-operatorie. Attualmente, circa il 39\% degli interventi chirurgici alla colonna vertebrale presso lo IOG non raggiunge un miglioramento clinicamente significativo, e il 22\% di questi è associato a esiti negativi. Queste cifre sono dovute, in parte, al fatto che l'istituto Galeazzi è una struttura terziaria che accoglie i casi più complessi provenienti da un vasto territorio. Inoltre, le problematiche vertebrali, specialmente in una popolazione che invecchia, rappresentano sfide terapeutiche considerevoli.\\

All'atto pratico Epimetheus rappresenta un'importante innovazione, in quanto utilizza i dati PROM raccolti direttamente dai pazienti per fornire predizioni accurate sul loro stato di salute a sei mesi dall'intervento. Queste previsioni possono guidare i medici nella personalizzazione delle cure post-operatorie, migliorando così gli esiti clinici e la qualità della vita dei pazienti.\\

L'interazione del medico con Epimetheus è successiva all'intervista al paziente, dove il primo sottopone il questionario al secondo prima della visita di controllo; qui vengono poste domande di autovalutazione del proprio stato di salute, le difficoltà nello svolgere determinate azioni nel quotidiano dopo l'operazione subita.

\section{PROM}
Quando si parla di PROM\footcite{site:utilizzo-prom-prem} si parla di misure che valutano lo stato di salute percepito direttamente dal paziente. I PROM indagano vari aspetti dello stato di salute del paziente, come la percezione dei sintomi, il dolore, l'ansia, la depressione e il grado di affaticamento. Questi strumenti aiutano a comprendere come i pazienti percepiscono la loro salute e il loro livello di disabilità, oltre a misurare la qualità della vita correlata alla salute. I PROM sono essenziali per raccogliere dati sulle condizioni di salute dei pazienti, che possono includere sintomi fisici e mentali, nonché il loro impatto sulla vita quotidiana.\\
I PREM (Patient-Reported Experience Measures) invece sono questionari che misurano la percezione dei pazienti riguardo alla loro esperienza durante le cure ricevute. Valutano aspetti come la qualità della comunicazione tra il paziente e il personale sanitario, il supporto ottenuto per la gestione di condizioni a lungo termine, il tempo trascorso in attesa di ricevere assistenza e la facilità di accesso alle cure. I PREM forniscono un feedback cruciale sull'esperienza complessiva del paziente nel sistema sanitario, permettendo di identificare aree di miglioramento nei servizi offerti.\\
L'adozione di PROM e PREM nei PSP offre diversi vantaggi significativi:
\begin{itemize}
    \item Valutazione del valore del PSP: questi strumenti permettono di dimostrare il valore del PSP in modo oggettivo agli specialisti sanitari. Monitorare l'andamento del trattamento attraverso PROM e PREM fornisce dati concreti che possono essere utilizzati per valutare l'efficacia del programma e migliorare la qualità delle cure fornite.
    \item Ottimizzazione dei servizi: PROM e PREM aiutano a ottimizzare i servizi offerti, consentendo di ridisegnare il percorso del paziente in base alle sue aspettative e necessità. Ad esempio, se i pazienti segnalano difficoltà nell'accesso alle cure o nella comunicazione con il personale sanitario, questi feedback possono essere utilizzati per apportare modifiche che migliorano l'esperienza del paziente.
    \item Personalizzazione dell'Assistenza: l'utilizzo di scale come la Morisky o l'ARMS permette di valutare l'aderenza del paziente al trattamento in modo oggettivo. Altre scale, come la PHE (Patient Health Engagement), utilizzate nei PSP di training, consentono di personalizzare l'assistenza in base al coinvolgimento del paziente nella gestione della propria terapia. Questo approccio centrato sul paziente assicura che le cure siano adattate alle specifiche esigenze e preferenze del paziente.
\end{itemize}


