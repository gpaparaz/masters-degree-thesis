\chapter{Sviluppi futuri}
\label{cap:sviluppi-futuri}

Poichè che l'approccio adottato è di tipo iterativo, sono state discusse delle linee guida per le possibili iterazioni future. Queste iterazioni mirano a migliorare ulteriormente l'efficacia e l'efficienza del sistema, sia dal lato backend che dal lato frontend.\\

Un importante passo avanti nel backend sarebbe la transizione dal salvataggio dei dati su file CSV a un sistema di database. L'attuale implementazione con file CSV è stata scelta per la sua rapidità di sviluppo e la sua capacità di fornire risultati immediati e sicuri. I file CSV, infatti, permettono di tenere traccia dei dati in modo semplice e immediato. Tuttavia, l'adozione di un database porterebbe a una gestione più strutturata e scalabile dei dati, facilitando operazioni di ricerca, aggiornamento e analisi.\\
Inoltre, tenere traccia della versione del software attualmente in uso consentirà di confrontare, in futuro, come i pareri dei medici riguardo alla piattaforma evolveranno in base agli avanzamenti del prodotto. Questo monitoraggio sarà essenziale per valutare l'impatto delle modifiche e per guidare lo sviluppo di nuove funzionalità.\\
Un altro sviluppo futuro riguarda il ripristino delle visualizzazioni relative alla spina dorsale. Attualmente, questo aspetto è limitato dall'obsolescenza delle librerie utilizzate, che sono state deprecate. Sarà necessario identificare nuove librerie che possano sostituirle, mantenendo o migliorando le funzionalità precedenti.\\

Un'evoluzione significativa è l'implementazione di un sistema di login per gli utenti. Questo permetterebbe di salvare i dati della sessione utente, migliorando l'esperienza e la personalizzazione del servizio. La gestione delle sessioni utente rappresenta un passo fondamentale verso un sistema più sicuro e strutturato.\\

Una richiesta iniziale dello stakeholder principale era quella di spostare maggiormente le logiche del wizard lato server. In particolare, è stato suggerito di ottenere il form relativo al paziente tramite una POST, dove il frontend comunica al backend le preferenze impostate dall'utente, e il backend risponde con l'intero form compilare. Sebbene questa specifica non sia stata implementata per motivi di tempo, la modularità del frontend rende questa integrazione facilmente realizzabile in futuro.\\

Un altro miglioramento auspicabile riguarda l'ampliamento delle visualizzazioni disponibili. Aumentare il numero e la varietà di visualizzazioni permetterebbe di avere un quadro sempre più chiaro dello stato di salute del paziente a seguito delle varie operazioni, migliorando la capacità dei medici di prendere decisioni informate.\\

Un'ulteriore ottimizzazione potrebbe essere l'introduzione di una doppia visualizzazione nella pagina dei risultati: una a griglia, simile a quella iniziale, e l'attuale a lista. La visualizzazione a griglia è più compatta e permette di vedere tutti i grafici con un colpo d'occhio, migliorando l'efficienza dell'interfaccia.\\

Molto interessante è stato il test condotto con il partecipante B (\ref{partecipante-b}) in quanto è emerso l'interesse per questo prodotto anche nell'ambito delle protesi audiometriche. Una implementazione di questo tipo richiederebbe sicuramente una base di machine learning differente per il backend, lato frontend invece sarebbe sufficiente inserire una pagina inziale dove si potrebbe differenziare il contesto di utilizzo, e mantenere all'interno dell'applicativo la stessa struttura dell'interfaccia. 

