\chapter{Introduzione}
\label{cap:introduzione}

% \noindent Esempio di utilizzo di un termine nel glossario \\
% \gls{api}. \\

% \noindent Esempio di citazione in linea \\
% \cite{site:agile-manifesto}. \\

% \noindent Esempio di citazione nel pie' di pagina \\
% citazione\footcite{womak:lean-thinking} \\

Al giorno d'oggi dobbiamo accettare i limiti della conoscenza e abituarci a convivere con l'incertezza, l'imprevisto e l'ignoto. Per lungo tempo, la natura e i suoi fenomeni erano avvolti da un'aura di mistero, gli eventi della vita, così come il dolore, la malattia e la morte, erano imprevedibili e incontrollabili, attribuiti agli umori e ai capricci degli dèi, spesso irascibili e permalosi. L'incertezza e gli imprevisti dominavano il mondo, e gli uomini, per affrontarli, si affidavano alla preghiera, al destino o alla buona sorte, accettandoli come parte della vita. Con il progresso, però, i successi straordinari della scienza ci hanno fatto credere che gli avvenimenti fossero prevedibili. Le leggi meccaniche di Newton, con la loro regolarità assoluta, hanno svelato i misteri dell'universo, permettendoci di prevedere il moto delle stelle, l'alzarsi delle maree o la traiettoria di un proiettile. Laplace ci ha persino suggerito che, conoscendo le condizioni iniziali, avremmo potuto ricostruire il passato e prevedere perfettamente il futuro, aprendo un dibattito filosofico ancora irrisolto su determinismo e libero arbitrio. Così, siamo passati rapidamente da un mondo incantato e imprevedibile a uno ordinato e preciso, regolato da leggi fisiche lineari e quindi facilmente prevedibili. La meccanica newtoniana, con la sua semplicità ed eleganza, si è identificata con il pensiero scientifico, estendendosi alle scienze biologiche e sociali, spesso con risultati deludenti.\\ 

Tuttavia, all'inizio del XX secolo, la scoperta della meccanica quantistica e lo studio dei sistemi complessi hanno messo tutto in discussione. Gli scienziati hanno compreso che molti fenomeni fisici, biologici e sociali sono governati dalle leggi della probabilità. In altre parole, gran parte di ciò che accade intorno a noi non può essere spiegato attraverso formule binarie, ma solo descritto mediante funzioni di probabilità. Questo ha segnato la fine dei dogmi, delle certezze assolute e ha sancito l'accettazione del dominio dell'incertezza nelle nostre vite.\\

Ciò che sorprende di più è che questa incertezza persiste anche con l'aumento dei dati disponibili per le decisioni. Abbiamo imparato che esiste una dissociazione incolmabile tra ciò che si può prevedere disponendo di molti dati e ciò che può essere utile per decidere nelle situazioni specifiche che ci riguardano. 
Con l'avvento del Machine Learning in ambito medico si è tentato di limitare l'incertezza, di fare predizioni quanto più precise possibili circa il futuro di un paziente a seguito di terapie o operazioni, ed è qui che entra in gioco Epimetheus.\\

\section{Obiettivi della tesi}
Epimetheus è una web application volta a fornire supporto nell'ambito delle decisioni mediche. Si tratta di un software realizzato su una base di Machine Learning che usa delle  predizioni per calcolare quale potrà essere lo stato di salute del paziente 6 mesi dopo un'operazione, e lo mostra all'utente attraverso l'uso di grafici. Le operazioni prese in esame sono quelle all'anca e ginocchio. Tale software verrà utilizzato presso l'ospedale Galeazzi di Milano; l'Istituto Ortopedico Galeazzi infatti è un grande ospedale universitario dove molti interventi invasivi sono affrontati con regolarità. Tuttavia non tutti gli interventi hanno come esito un miglioramento clinico, e il 22\% di questi hanno addirittura un peggioramento. \\
Attraverso la data science, quindi, Epimetheus vuole svolgere la funzione di supporto al medico affinchè possa stabilire se un'operazione avrà esito positivo. Ciò avrà come conseguenza ridurre gravosi interventi a pazienti che non ne avrebbero giovamento, ridurre le spese sostenute per trattamenti non necessari, e ottimizzazione delle risorse. \\

L'obiettivo di questa tesi è la trasformazione di Epimetheus da \gls{poc} a software scalabile; allo stesso tempo è stato operato un intervento sulla user experience per rendere l'applicazione più fruibile all'utente inesperto.

