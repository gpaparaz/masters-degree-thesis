\chapter{Metodologia}
\label{cap:metodologia}

% \intro{Breve introduzione al capitolo}\\

\section{Metodologia agile}
\label{sec:metodologia-agile}

Gli sviluppi si sono svolti seguendo la metodologia agile\footcite{womak:agile-manifesto}: sono state schedulate due riunioni mensili allo scopo di tenersi aggiornati sugli sviluppi in corso, in assenza di riunioni c'è comunque stato scambio di messaggi per mantenere presente e continua la comunicazione.\\
Il principio di fondo che si è seguito è quello dello sviluppo iterativo, ovvero tenendo presente l'obiettivo finale del software si è proceduto per step implementando prima soluzioni più semplici ed immediate e poi si è proceduto nell'evoluzione di alcune funzionalità per implementare soluzioni più performanti e scalabili. \\

Le comunicazioni all'interno del team si sono basate su riunioni online, di persona e condivisione di materiale informativo per accordarci sul lavoro da svolgere.\\
Il materiale usato è stato condivisione di documenti drive e di file Figma, nello specifico parte integrante dell'attività di comunicazione è stata l'uso dell' EventStorming\footcite{womak:event-storming}.\\
L'EventStorming è una tecnica collaborativa utilizzata principalmente per esplorare e modellare processi complessi e sistemi software attraverso la narrazione di eventi che accadono all'interno di un sistema. È stata sviluppata da Alberto Brandolini nel 2013 ed è particolarmente utile nel contesto del Domain-Driven Design (DDD). L'obiettivo principale di EventStorming è di ottenere una comprensione condivisa del dominio tra tutti i partecipanti, che può includere sviluppatori, stakeholder, esperti di dominio e altri membri del team.

\begin{figure}[!ht] 
    \centering 
    \includegraphics[width=0.9\columnwidth]{event-storming} 
    \caption{Esempio di workflow utilizzato per Epimetheus}
\end{figure}

\subsection{Componenti e fasi dell'EventStorming}
\begin{itemize}
    \item \textbf{Event:} Gli eventi sono i cambiamenti di stato significativi che accadono all'interno del sistema. Vengono rappresentati come note adesive arancioni e poste su una lavagna o un muro in ordine cronologico. Gli eventi sono descritti in modo semplice, solitamente come frasi al passato (e.g., "Ordine confermato", "Pagamento ricevuto").
    \item \textbf{Timeline:} Gli eventi vengono organizzati lungo una linea temporale per rappresentare la sequenza in cui accadono. Questo aiuta a visualizzare il flusso di processi e a identificare eventuali lacune o incongruenze.
    \item \textbf{Attori:} Gli attori sono le persone o i sistemi che causano gli eventi. Questi vengono spesso rappresentati con note adesive di colore diverso, collegati agli eventi che generano.
\item \textbf{Comandi:} I comandi sono azioni o intenzioni che provocano eventi. Vengono rappresentati con note adesive blu e posizionati prima degli eventi che innescano.
\item \textbf{Aggregati:} Gli aggregati sono cluster di eventi e comandi che rappresentano unità logiche all'interno del dominio. Vengono spesso identificati in una fase successiva per semplificare e strutturare il modello.
\item \textbf{Polizze e regole:} Le regole di business o le politiche che influenzano come e quando gli eventi accadono. Possono essere rappresentate con note adesive di un altro colore
\end{itemize}

\subsection{Tipi di EventStorming}
\begin{itemize}
    \item \textbf{Big Picture EventStorming:} Questa variante è utilizzata per avere una visione d'insieme del dominio. Coinvolge un gran numero di partecipanti e mira a identificare tutti gli eventi rilevanti, i punti di interesse e le problematiche principali.
    \item \textbf{Process Modeling EventStorming:} Questa variante si concentra su specifici processi o flussi all'interno del dominio. È più dettagliata e mira a modellare con precisione come funzionano questi processi.
    \item \textbf{Design-Level EventStorming:} Questa variante è utilizzata per progettare soluzioni tecniche specifiche. Coinvolge principalmente sviluppatori e si concentra su come implementare gli eventi e i comandi nel codice.
\end{itemize}

\subsection{Benefici dell'EventStorming}
\begin{itemize}
    \item \textbf{Comprensione condivisa:} Favorisce la collaborazione tra tutti i partecipanti, aiutando a costruire una comprensione comune del dominio
    \item \textbf{Identificazione delle lacune:} Permette di visualizzare facilmente le aree del sistema che potrebbero essere poco chiare o incomplete
    \item \textbf{Coinvolgimento degli esperti di dominio:} Gli esperti di dominio possono contribuire direttamente alla modellazione, assicurando che il sistema riflette accuratamente la realtà del business.
    \item \textbf{Velocità e flessibilità:} La natura visiva e interattiva dell'EventStorming lo rende un metodo rapido per esplorare e modellare domini complessi rispetto ai metodi tradizionali di documentazione.
\end{itemize}

\section{Sviluppo iterativo}
\label{sec:sviluppo-iterativo}

\section{Event storming}
\label{sec:event-storming}

% Di seguito viene data una panoramica delle tecnologie e strumenti utilizzati.

% \subsection*{Tecnologia 1}
% Descrizione Tecnologia 1.

% \subsection*{Tecnologia 2}
% Descrizione Tecnologia 2

% \section{Ciclo di vita del software}
% \label{sec:ciclo-vita-software}

% \section{Progettazione}
% \label{sec:progettazione}

% \subsubsection{Namespace 1} %**************************
% Descrizione namespace 1.

% \begin{namespacedesc}
%     \classdesc{Classe 1}{Descrizione classe 1}
%     \classdesc{Classe 2}{Descrizione classe 2}
% \end{namespacedesc}


% \section{Design Pattern utilizzati}

% \section{Codifica}
